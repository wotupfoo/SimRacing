To determine the force required to rotate the mass by 20 degrees in 250 ms, we'll break down the problem into parts:

1. **Calculate the required angular acceleration**.
2. **Determine the torque needed** to achieve this angular acceleration.
3. **Calculate the force needed** based on the given torque and distance from the pivot.

### Step 1: Calculate the Required Angular Acceleration

The mass needs to rotate by 20 degrees (which is about \( \theta = 0.349 \) radians) in 250 ms (or 0.25 s). Assuming it starts from rest and accelerates uniformly, we can use the rotational kinematic equation:

\[
\theta = \frac{1}{2} \alpha t^2
\]

Solving for \( \alpha \) (angular acceleration):

\[
\alpha = \frac{2\theta}{t^2}
\]

### Step 2: Determine the Torque Needed

Torque (\( \tau \)) is related to angular acceleration by the moment of inertia (\( I \)):

\[
\tau = I \alpha
\]

Since the mass can be considered as a point mass at a distance \( r = 0.25 \, \text{m} \) from the pivot, we can approximate its moment of inertia as:

\[
I = m r^2
\]

### Step 3: Calculate the Force Needed

Torque (\( \tau \)) is also related to the applied force (\( F \)) and the distance from the pivot at which it acts (\( d = 0.5 \, \text{m} \)):

\[
\tau = F \times d
\]

From this, we can rearrange to solve for \( F \):

\[
F = \frac{\tau}{d}
\]

Let’s go through these steps with the values you’ve provided.

The required force to rotate the mass by 20 degrees in 250 ms is approximately **279.25 N**.